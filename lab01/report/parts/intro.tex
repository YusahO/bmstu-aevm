\chapter*{Цель работы}
\addcontentsline{toc}{chapter}{Цель работы}

Основной целью работы является ознакомление с принципами 
функцио\-нирования, построения и особенностями архитектуры 
суперскалярных кон\-вейерных микропроцессоров.

Дополнительной целью работы является знакомство с принципами 
про\-ектирования и верификации сложных цифровых устройств с использованием
языка описания аппаратуры SystemVerilog и ПЛИС.

\chapter{Основные сведения}

RISC-V является открытым современным набором команд, кото\-рый может 
использоваться для построения как микроконтроллеров, так и 
высоко\-производительных микропроцессоров. Таким образом, термин RISC-V 
фак\-тически является названием для семейства различных систем команд, 
кото\-рые строятся вокруг базового набора команд, путем внесения в него
различ\-ных расширений.

В данной работе исследуется набор команд RV32I, который включает в себя 
основные команды 32-битной целочисленной арифметики кроме умно\-жения и 
деления.

\section{Модель памяти}
Архитектура RV32I предполагает плоское линейное 32-х битное адрес\-ное 
пространство. Минимальной адресуемой единицей информации является 1 байт. 
Используется порядок байтов от младшего к старшему (Little Endian), 
то есть, младший байт 32-х битного слова находится по младшему адресу 
(по смещению 0). Отсутствует разделение на адресные пространства команд, 
данных и ввода-вывода. Распределение областей памяти между различными 
устройствами (ОЗУ, ПЗУ, устройства ввода-вывода) определяется реализа\-цией.

\section{Система команд}
Большая часть команд RV32I является трехадресными, выполня\-ющими операции 
над двумя заданными явно операндами, и сохраняющими результат в регистре. 
Операндами могут являться регистры или константы, явно задан\-ные в коде 
команды. Операнды всех команд задаются явно.

Архитектура RV32I, как и большая часть RISC-архитектур, предпо\-лагает
разделение команд на команды доступа к памяти (чтение данных из памяти 
в регистр или запись данных из регистра в память) и команды обработки 
данных в регистрах.
